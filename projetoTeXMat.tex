\documentclass[a4paper, 12pt]{article}
\usepackage[T1]{fontenc}
\usepackage[utf8]{inputenc}
\usepackage[brazilian]{babel}
\usepackage{indentfirst}
\usepackage[top = 2cm, bottom = 2cm, right = 2.5cm, left = 2.5cm]{geometry}
\linespread{1.3}
\usepackage{amsmath, array, amssymb} % esse pacote permite a criação de matrizes de forma mais rápida e prática
% portanto, podemos utilizar os pacotes acima para gerar o ambiente 'pmatrix'
% diferentemente da criação de matrizes pelo ambiente array, com o ambiente pmatrix não precisamos de left/right para completar o código


\title{{\LARGE \textbf{Exercícios de Fixação - Curso LaTeX}}}
\author{Autor: Kauê Scaranari Alcântara}
\date{Data: \today}

\begin{document}
\maketitle
Seguem os exercícios resolvidos referentes à criação de expressões matemáticas:

Exercício 1:
\begin{equation}
\sqrt[3]{\left(\frac{2^{3} + 2^{5}}{10}\right)}
\end{equation}

Exercício 2:
\begin{equation}
\overline{(x \cdot y)^{4}} = \overline{x^{4}} \cdot \overline{y^{4}}
\end{equation}

Exercício 3:
\begin{equation}
\frac{a}{\sin\widehat{A}} = \frac{b}{\sin\widehat{b}} = \frac{c}{\sin\widehat{C}} = 2r
\end{equation}

Exercício 4:
\begin{equation}
\|\vec{u} \times \vec{v}\| = \|\vec{u}\| \cdot \|\vec{v}\| \cdot \sin\left(\theta\right)
\end{equation}

Exercício 5:
\begin{equation}
\frac{1}{\left(\frac{2}{3}\, cm/s \right)^{2}} \dfrac{\partial^{2}\Psi}{\partial t^{2}} - \dfrac{\partial^{2}\Psi}{\partial x^{2}} = 0
\end{equation}

\vspace{0.5cm}

Abaixo, encontra-se um extenso ambiente matemático para estudo.
Podemos produzir um texto que apresenta expressões matemática no decorrer de sua leitura. Como exemplo, temos: $x = 2$, ou seja, $x$ está valendo 2.
Abaixo utilizamos o ambiente equation da seguinte forma:
\begin{equation}
x = 2.
\end{equation}

Os operadores básicos são \textbf{soma}, \textbf{subtração}, \textbf{multiplicação} e \textbf{divisão}
\begin{itemize}
\item Para soma: $x = 2 + 1$
\item Para subtração: $x = 2 - 1$
\item Para multiplicação: $x = 3*6$ ou $x = 3.6$ ou $x = 3 \times 6$
\item Para divisão: $x = 6/3$
\item Para frações: $\frac{6}{2}$ ou também $\frac{4+3+2+1}{6\times8}$
\item Para exponenciação: $a^b$, $a^{b+c}$, $a^{\frac{b}{c}}$
\item Para radiciação: $\sqrt[a]{b}$ ---> Lê-se raíz a de b, $\sqrt[2]{4}, \sqrt{81}$
\item Para logaritmo: $\log 4$ Lê-se logaritmo de 4, $\log_{2}4$ Lê-se logaritmo de 4 na base 2
\item Para seno: $\sin 60º$
\item Para cosseno: $\cos 60º$
\item Para tangente: $\tan 60º$
\end{itemize}

Agora podemos aplicar as propriedades aprendidas anteriormente às funções:
\begin{itemize}
\item $f(x) = 2x + 2$
\item $f(x) = 2\cos x$
\item $f(x) = 2x^{2} + x + 4$
\item $f(x) = \sqrt[3]{x} + 2x + x^{2}$
\item $f(x) = 2x^{2} + x + 4$
\end{itemize}

O ambiente matemático do LaTeX também é útil para que possamos escrever reações químicas.\\

Vejamos como exemplo uma reação básica para a fotossíntese:\\
\begin{center}
$6 CO_{2} + 6 H_{2}O + calor \rightarrow 6 O_{2} + C_{6}H_{12}O_{6}$
\end{center}

Para o cálculo de limites:

\begin{equation}
lim_{x \rightarrow 1}(x^{3} - 3)
\end{equation}
\vspace{0.5cm}
\begin{equation}
lim_{x \rightarrow 2}\sqrt{x^{4}-8}
\end{equation}
\vspace{0.5cm}
\begin{equation}
\lim_{x \rightarrow -3}\left(\frac{x^{2}-9}{x+3}\right)
\end{equation}
\vspace{0.5cm}
\begin{equation}
\lim_{x \rightarrow \infty}\frac{1}{x}
\end{equation}
\vspace{0.5cm}
\begin{equation}
\int(e^{-x} + 2^{x})dx 
\end{equation}
\vspace{0.5cm}
\begin{equation}
\int_{a}^{b}f(x)dx = F(b) - F(a)
\end{equation}\\
\vspace{0.5cm}
Em seguida, temos a equação da Lei da Gravitação Universal de Isaac Newton:
\begin{equation}
\vec{F} = -G\frac{m_{1}m_{2}}{r^{2}}\hat{r},\hspace{1cm} G = 6.6 \times 10^{-11}\frac{m^{3}}{Kg^{-1}s^{-2}}
\end{equation}
\vspace{0.5cm}
Aplicamos agora a construção de uma equação em t diferenciada:
\begin{equation}
f(t) = \frac{1}{2} + \frac{\cos\frac{\pi}{3}}{2\pi}\sum_{-\infty}^{\infty}\frac{1}{n}e^{Bn2\pi t}
\end{equation}
\vspace{0.5cm}
\begin{equation}
\frac{a}{b+\frac{b+1}{c+\frac{c+1}{d+\frac{d+1}{e}}}}
\end{equation}
\vspace{0.5cm}
Agora vejamos os limitadores left/right na prática que servem para demarcar uma equação/expressão:
\begin{itemize}
\item $\left(\frac{a}{b}\right)$
\item $\left\lbrace\frac{a}{b}\right\rbrace$
\item $\left[\frac{a}{b}\right]$
\end{itemize}
\vspace{0.5cm}
Vamos criar agora na prática mais um exemplo de equação:
\vspace{0.3cm}
\begin{equation}
\frac{d}{dt}\left(mr^{2}\frac{d\theta}{dt}\right) = 0
\end{equation}

\vspace{0.5cm}

\begin{equation}
\left(
\begin{array}{lr}
	a & b \\
	c & d \\
\end{array}
\right)
\end{equation}

\vspace{0.5cm}

\begin{equation}
\left(
\begin{array}{lcr}
	a & b & c \\
	d & e & f \\
	g & h & i \\
\end{array}
\right)
\end{equation}
\vspace{0.5cm}
\begin{equation}
\begin{pmatrix}
a & b & c \\
d & e & f \\
\end{pmatrix}
\end{equation}
\vspace{0.5cm}
\begin{equation}
\begin{pmatrix}
a & b & c & d \\
e & f & g & h \\
i & j & k & l \\
m & n & o & p \\
\end{pmatrix}_{4 \times 4}
\end{equation}

\vspace{0.5cm}

\begin{equation}
	\left\lbrace
	\begin{array}{cc}
		3x + 2y = 6 \\
		2x + 3y = 5 \\
	\end{array}
	\right.
\end{equation}

\vspace{0.5cm}

\begin{equation}
	\left\lbrace
	\begin{array}{ccc}
		x + y + z = 6\\
		x + 2y + 2z = 9\\
		2x + y + 3z = 11\\
	\end{array}
	\right.
\end{equation}

\vspace{0.5cm}

% Podemos escrever sistemas lineares usando o ambiente 'eqnarray', que representa um ambiente para equações de arrays (não há chaves nesse ambiente)

% sistema linear com duas linhas - usando o ambient eqnarray não precisamos iniciar o begin{equation}
\begin{eqnarray}
	3x + 2y = 6 \\
	2x + 3y = 5 
\end{eqnarray}

\vspace{0.5cm}

\begin{eqnarray}
	x + y + z = 6 \\
	x + 2y + 2z = 9 \\
	2x + y + 3z = 11
\end{eqnarray}

\vspace{0.5cm}

\begin{equation}
f(x) =
\left\lbrace
\begin{array}{cc}
	x - 1,& x = 2\\
	2x + 3,& x \neq 2\\
\end{array}
\right.
\end{equation}

\vspace{0.5cm}

\begin{equation}
det(A) =
\left\vert
\begin{array}{lr}
	a & b \\
	c & d \\
\end{array}
\right\vert
\end{equation}

\vspace{0.5cm}

\begin{equation}
det(B) =
\left\vert
\begin{array}{lcr}
	a_{11} & b_{12} & c_{13} \\
	d_{21} & e_{22} & f_{23} \\
	g_{31} & h_{32} & i_{33} \\
\end{array}
\right\vert
\end{equation}

\vspace{0.5cm}

Temos agora, portanto, uma opção para escrever expressões com sobrescritos e subscritos:\\

% $\stackrel{}{}$

Exemplo: $A + B\, \stackrel{2\,min}{\longrightarrow}\, C + D$

\end{document}